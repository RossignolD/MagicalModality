\chapter[Kripke frames and models]{Sing a song of Kripke: Kripke frames and models}

Two of the most fundamental objects of study in Modal Logic are Kripke frames and Kripke models, both named for Saul Aaron Kripke,
(1940--2022). My lecturer in Amsterdam, who taught me the mathematical foundations of the field of Modal Logic, always
said that there were two kinds of Modal Logicians: those who prefer to think in terms of Kripke frames, and those who prefer
to think in terms of Kripke models. I suppose that I am more of a Kripke model person myself, despite not really 
being a model theorist myself. You will see that the two objects roughly correspond to a syntactic (proof theoretic) and
semantic (model theoretic) position, respectively. I will leave that matter of taste to you.

\setcounter{section}{-1}

\section{The bare definitions}

A \emph{Kripke frame} (also known as a relational frame, among other names) is an ordered pair $\mathfrak{F} = (W, R)$ that consists of
a set of possible worlds $W$ and an accessibility relation between them denoted $R$. If we are discussing multiple different
frames ($\mathfrak{F}$ and $\mathfrak{G}$ perhaps), we might choose to distinguish their sets of possible worlds as $W^\mathfrak{F}$
and $W^\mathfrak{G}$ respectively. And similarly, for their accessibility relations, we may denote the two relations as
$R^{\mathfrak{F}}$ and $R^{\mathfrak{G}}$.

A \emph{Kripke model} (also known as a relational model, possible--worlds model, among other names) is an ordered trirple
$\mathfrak{M} = (W, R, V)$ that consists of a set of possible worlds $W$, an accessibility relation $R$, and a valuation function
$V$. The valuation assigns propositions such as $P$ truth values (either true or false) in worlds such as $\alpha$. Another way
of thinking about this is that $V$ tells which possible worlds have which propositions true in them, therefore assigning propositions
to worlds or sets of worlds. Another way to think about Kripke models is that they are Kripke frames that come with
a truth function $V$. In fact, the defintion that I was taught was precisely that: a Kripke model is an ordered pair $ \mathfrak{M}=
(\mathfrak{F}, V)$ where $\mathfrak{F}$ is a Kripke frame and $V$ is a valuation function.

\subsection*{Some quick notation notes}


The accessibility relation $R$ tells us how we can ``get from'' one possible world in the Kripke frame to another. If 
$(\alpha, \beta) \in R$, then we can say that ``$\alpha$ sees $\beta$'', or ``we can reach $\beta$ from $\alpha$''. The accessibility
relation for a Kripke frame is usually written $R(\alpha, \beta)$ or as simply $R \alpha \beta$.

When a proposition $P$ is true in a certain possible world $\alpha$ in a Kripke model, we say that ``$P$ is true at $\alpha$'',
``$\alpha$ satisfies $P$'', or, in certain texts, ``$\alpha$ forces $P$''. These all mean the same thing. They are written
with the so--called \emph{modal turnstile} symbol $\Vdash$. The symbol is an infix, so its correct usage is something like
$\alpha \Vdash P$ in the case that $\alpha$ satisfies $P$. Some people like to use the \emph{semantic turnstile} symbol 
$\models$ for this, reflecting the fact that we are asserting something about the semantics of a proposition. I think that the
distinction between satisfaction in a Kripke model and satisfaction in general is important, so I will stay away from overloading
the $\models$ symbol. Similarly to the ``factions'' that have arisen over whether the more primitive object is the Kripke
frame or model, there are ``factions'' that ``argue'' about whether to use $\Vdash$ or $\models$ for modal satisfaction.
Ultimately, this is a matter of taste.

\section{A fast overview of the semantics of the symbols}
You've already seen this in Chapter 2 of \emph{Modal Logic for Open Minds} as Definition 2.2.2, but I will give it here anyway
so that you don't have to go hunting for it the next time you need it. These are the truth--definitions of the operators
as they behave in Kripke models. So here we go! For a Kripke model $\mathfrak{M}$ and a possible world $\alpha$ of that 
Kripke model, we say that \dots
\begin{itemize}
    \item $\alpha \Vdash P \IFF$ the valuation operator $V$ assigns True to $P$. 
    \item $\alpha \Vdash \logicnot \phi \IFF \NOT (\alpha \Vdash \phi)$.
    \item $\alpha \Vdash \phi \wedge \psi \IFF \alpha \Vdash \phi \AND \alpha \Vdash \psi$.
    \item $\alpha \Vdash \logicdiamond \phi \IFF \bigvee \beta$ such that $R \alpha \beta$, $\beta \Vdash \phi$.
    \item $\alpha \Vdash \logicbox \phi \IFF \bigwedge \beta$ such that $R \alpha \beta$, $\beta \Vdash \phi$.
\end{itemize}

The rest of the logical connectives' semantics can be deduced from these, as $\logicnot$ and $\wedge$ are a functionally 
complete set. I will leave it as an exercise for you to complete if you desire.

\section{Modal axioms, what are they good for?}
Modal axioms are formulae in the language of modal logic that are true for Kripke frames that have a particular structure
to their accessibility relation. For example, in every Kripke frame that has a reflexive accessibility relation,
it is the case that the Kripke model has the statement $\logicbox \phi \rightarrow \phi$ true. And the converse of this 
statement is true as well: Every Kripke frame for which the statement $\logicbox \phi \rightarrow \phi$ holds at every possible
world has a reflexive accessibility relation. We can prove this using a picture. 

We will proceed by contradiction. Assume for the sake of contradiction that the axiom $\logicbox \phi \rightarrow \phi$
is false in some possible world $\alpha$ in our Kripke frame $\mathfrak{F}$. We will show this as the following
(the large bubble is $\alpha$):

\begin{center}
    \includesvg{Tproof1.svg}
\end{center}

We will use the principle of classical logic that $\logicnot(\phi \rightarrow \psi) \leftrightarrow \phi \wedge \logicnot \psi$
to expand the assumption.

\begin{center}
    \includesvg{Tproof2.svg}
\end{center}

Separating the two conjuncts and applying the semantics of the box symbol means that every possible world that $\alpha$ can see
satisfies $\phi$. 

\begin{center}
    \includesvg{Tproof3.svg}
\end{center}

But if our accessibility relation is reflexive, then every possible world can see itself. Then $\phi$ must be true in $\alpha$
for this reason. 

\begin{center}
    \includesvg{Tproof4.svg}
\end{center}

A contradiction arises from the fact that we already asserted that $\logicnot \phi$ is true in this world
when we separated the two conjuncts. Thus we cannot have a Kripke frame that contains this statement and is not reflexive.

\begin{center}
    \includesvg{Tproof5.svg}
\end{center}

I will fully accept this kind of ``picture proof''
from you as a demonstration of these types of theorems, most notably because proving these things formally (in a natural
deduction system perhaps), is not the goal of this project. I have written formal proofs of fairly simple correspondences of 
this type that involved 5 layers of nested subderivations, which I do not believe would give you more enlightenment into 
the axioms than the ``picture proofs''.

You may have noticed that we used the reasoning principle of \emph{modus ponens} in this proof. Without this essential piece
of reasoning, logic has a hard time getting started. So we build this into every Kripke frame through the following axiom:
$\logicbox(\phi \rightarrow \psi) \rightarrow (\logicbox \phi \rightarrow \logicbox \psi$. This axiom is so foundational,
it was given the initial \textbf{K} in honor of Kripke. Nearly all modal axioms have initials. The one that we just discussed
and proved a theorem about is called \textbf{T}. We will continue to explore modal axioms and their relation to Kripke frames
and models as a theme in this project.

\section{Looking into next week}
For next week, I would like you to hear van Benthem's take on the ``landscape of modal logics'' by reading Chapter 8 in 
the book. Also, if you have gotten down to here, you should have read through my notes here thoroughly. I invite you to 
attempt his Exericse 2 in that chapter. And just for fun, becuase I think that you should do this to get acquainted with 
the way that modal axioms play with the accessibility relations on their Kripke frames, you should go to the Wikipedia
page for the Kripke semantics and try to prove at least two of the correspondences shown in the table \emph{Common
modal axiom schemata}. Don't worry about reading the entire page and understanding it. The correspondences are what we will
be building upon in the course of the project.