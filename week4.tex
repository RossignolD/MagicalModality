\chapter[A taste of some applications of modal logic]{Speed--dating for applications of modal logic}

Modal logic finds its way into many fields of study, from philosophy to computer science, to mathematics,
and beyond. I will quickly summarise these applications. I will put an asterisk next
to sections on applications that I have personally seen. Please look at all of the
applications I present. I will let each of you choose an application that excites or
interests you.

\setcounter{section}{-1}

\section{Deontic modal logic*: the logic of preference and responibility}
Deontic modal logic comes from the idea of trying to express ideas from the field of ethics and aesthetics
in the language of logic. Asking questions like, ``If something is obligatory, is it necessarily permissible?''
is core to this branch of modal logic. This type of ``ought implies can'' statement has been analyzed
throughout the history of philosophy, most notably by Kant. It has only been with the last century that modal 
logic has been used to attempt to give meaning to these statements in formal logical frameworks.

Often, the box and diamond symbols are replaced with $O$ and $P$ (standing for \emph{obligatory} and
\emph{permissible}, respectively.) In these forms, the accessibility relation in a Kripke frame 
or model is interpreted as saying that if $\alpha$ and $\beta$ are possible worlds that have $R \alpha \beta$,
then $\beta$ is \emph{acceptable} to $\alpha$ IFF all obligations fulfilled by $\alpha$ are fulfilled
in $\beta$. Thus, this logic has the axiom system known as \textbf{KD}, sometimes called \textbf{D} 
(because the \textbf{K} axiom is assumed to be true in all modal logic systems of axioms).

Deontic logic and its axiom \textbf{D}, are named from the Greek word \emph{d\'eon}, meaning ``that thing
that is proper.''

\section{Epistemic logic: what can we say if we don't know all the facts?}

Epistemic logic concerns itself with what we know and how we come to know those things. For example, if it is the case that Alice knows
something, is it the case that she knows that she knows it? If this is the case, it would be written in the language of modal logic as
$\logicbox_A \phi \rightarrow \logicbox_A \logicbox_A \phi$, where the symbol $\logicbox_x \phi$ is read as ``$x$ knows $\phi$''. The diamond
symbol $\logicdiamond_x \phi$ is sometimes interpreted as ``$\phi$ is consistent with $x$'s knowledge'. There are other possible interpretations.

Some logicians like to add another pair of operators into the mix that represent belief and consistency with belief. These logicians are likely
to write $K_x \phi$ for the $\logicbox_x$ symbol, and use the duality of the operators to define the $\logicdiamond_x$ symbol as $\logicnot K_x
\logicnot \phi$. The operators for belief then look like $B_x \phi$ for the idea that $x$ believes that $\phi$, and $\logicnot B_x \logicnot \phi$,
for the dual in an analogous fashion. The ways that these modalities interact is the interesting part. This is clear from the fact that
by the nature of knowledge and belief that most other philosophers assert, any Kripke model that represents knowledge and belief must bear the
axiom system \textbf{S5}, which comes with the axioms \textbf{K}, \textbf{T}, and \textbf{4} and \textbf{B}. Since this axiom set produces an
accessibility relation that is an equivalence relation, there is much to be said here about what exactly the accessibility of one world to another
represents. 

Epistemic logic gets its name from the same Greek word as epistemology: \emph{epithasthai} meaning ``to know'' turned into a noun, \emph{epist\=em\=e},
meaning ``knowledge''.

\section{Information dynamics: when we don't know what everyone knows}

Information dynamics, a favorite topic of van Benthem, concerns how knowledge is transferred between groups or individuals. The classic logic
riddles about people wearing different colored hats can be solved using the types of Kripke models built by logicians who study information
dynamics. Similarly, the puzzle of ``Cheryl's birthday'' that was viral on the internet once, can be solved in this way. One of the prominent
logical systems used to model this type of logic is the Public Announcement Logic (PAL). If I say to Alice and Bob, ``I like to eat chocolate
ice cream after dinner,'' I likely said this not knowing what the ``state''of your knowledge of my ice cream preferences. But now, I can make some
definitive and complex epistemic statments like $\logicbox_{Mei} \logicbox_{Alice} \phi$. I can even say something like $\logicbox_{Mei} \logicbox_{Alice}
\logicbox_{Bob} \phi$. These statements, iterated through each other into the limit, seem to capture the idea of ``common knowledge''. 

After the PAL had been around sufficiently long in the information dynamics community, it found itself a generalization: the Dynamic Epistemic
Logic (DEL). The DEL can express all of the power of the PAL, but allows updates to the Kripke models over time as more knowledge and individuals
are added. It can also be used to model scenarios where individuals are aware that their beliefs can be false.

\section{Modal logic and linguistics*: how do natural languages capture our ideas about necessity?}

This application lies somewhere between all of the others above and the usual translation of the box and diamond. The linguistic questions that are
raised by modal logic cut accross the different philosophical topics above and into the heart of ``What does it mean for something to be necessary
or possible?'' Another question is, ``How do languages express these ideas, and what subtleties are there?'' An example from English is the
following: in certain regional varieties of English, the sentence ``I might could come to dinner tomorrow.'' is perfectly valid. A speaker
of this variety of English would tell you that it means ``It is possible that I will come to dinner tomorrow.'' So what is the semantic purpose
of the second modal word here? Is it possible to interpret this type of shift of meaning in a Kripke frame?

Other questions involve the subtlety of ``should'' vs. ``must''. I always got myself in trouble in German class because I couldn't distinguish
when to say \emph{Ich muss} and when to say \emph{Ich sollte}. The two meanings are ``I must'' and ``I should'', respectively. These both seem
as though the $\logicbox$ symbol will be used. But does this alone capture what the difference between these words? 

\section{Modal logic and computer science: what does a modal logic fomula mean to a computer?}
An application of modal logic that has been becoming popular in the software development communities, known as formal verification, is based
around using formal proofs in logical systems to ``prove'' that the software is free of all bugs. The fact that this can be done at all
is owed to the Curry--Howard correspondence. A common way to do this type of proof is to constrain programs to those that can be represented
by modal formulae. The catch here is that a more complicated but robust framework for modal logic must be used. Often this is the Propositional
Dyanmic Logic (PDL).

The advantage of the PDL is that it has both \emph{atomic formualae} and \emph{atomic programs}. This allows us to discuss states of programs
as well as outputs. A modal symbol for each program $\phi$ looks like $[\phi]A$. This is read as ``$A$ is true after program $\phi$ has run.''
A new set of symbols to deal with program failure, concatenation of programs, and non--deterministic programs are introduced in addition to all
of these modal symbols.

\section{Provability*: when we can't have everything we want}
The logic of provability allows us to connect modal logic to deep themes in mathematical logic, such as the soundness and completness of Peano arithmetic,
nd the Incompleteness results. Here, the symbol $\logicbox \phi$ is read as ``$\phi$ is provable''. Under this interpretation, the 
\textbf{L} axiom, named for M. L\:{o}b, takes on an interesting meaning. This axiom, which states that $\logicbox (\logicbox \phi \rightarrow
\phi) \rightarrow \logicbox \phi$, seems to contain itself in a self--referential way. And indeed, it is this self--reference that gives power
to the system of axioms containing it. Seemingly out of nowhere, the system \textbf{GL}, containing only the \textbf{K} and \textbf{L} axioms,
gives us a route to the Incompleteness results.

This idea of proving things about what is provable is quite magical, in the opinion of van Benthem. Indeed, this is why I named the project as I did.
Even the mathematical proof theory of \textbf{GL} is subtle but rich, as it admits a cut--elimination proof in the appropriate sequent calculus.
This is far from an insignificant consequence.

\section{Until we meet again\dots}

Please read these little ``blurbs'' on the different applications and
let me know which of the applications you are interested in by Wednesday of the
week. That way, I can begin to plan and find resources if it is not covered in van
Benthem's book. In terms of the order in which you will receive your chosen application as the ``Application of the Week'',
I will proceed in the order in which I receive emails from each of you. 