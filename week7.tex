\chapter[Week of Deontic Logic]{Better to ask forgiveness: What can Deontic Logic tell us about obligation and permissibility?}

Deontic Logic has been one of my ``side-interests'' in the landscape of Modal Logic applications, mostly because I find
its intersections with linguistics and the broader scope of philosophy cool. When I did my first round of applications to
graduate school, I applied to many linguistics programs hoping to work on something like this. I wanted to know if the
ways that languages express necessity and possibility lined up in an analogous way to their expressions of obligation
and permissibility. After all, we treat these two pairs as equivalent in Deontic Logic. It would seem that this works
in English, as \emph{not possibly not $\phi$} and \emph{not permissible not $\phi$} do seem equivalent to
\emph{necessarily $\phi$} and \emph{exists an obligation to $\phi$} to me. I wanted to see if these hold in other languages, and
even to see if a larger sample of English speakers felt as I do about these modalities.

\setcounter{section}{-1}

\section{Your mission, should you choose to accept it...}

Van Benthem relates Deontic Logic to the logic of preferences in his book in an interesting way, but he does it in a very
formal structure that you may find overkill on the subject, losing the forest for the trees. If you want to see his
reasoning, I would recommend it after reading some of the first chapter of \emph{The Handbook of Deontic Logic and Normative
Systems}. Here is the url for the Handbook: \url{http://www.collegepublications.co.uk/downloads/handbooks00001.pdf}.

\bigskip

Yes, I know that the chapter is ~45 pages long. I think that it gives a sufficient number of the historical problems
that motivated the creation of Deontic Logic. I am not assigning any exercises this week, so do allow enough time to read
the chapter here and possibly Chapter 16 of \emph{Modal Logic for Open Minds}.