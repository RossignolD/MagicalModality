\chapter[Week of Information Dynamics]{What you don't know will help you: An introduction to Information Dynamics}

This application of Modal Logic, in my opinion, is a little bit like a magic trick, in the sense that it produces spectacular solutions
to problems that seem to come from nowhere. But of course, what is really happening is that there are deductive steps occurring behind
the scenes, so what we ``see'' is just the surface level of the problem being solved. That is, until the dedicated Logician decides
to dive deeper. A wise lecturer once chided me for saying that a certain topic (Pythagorean triples on the complex plane) was too
elementary to study as an  independent study. He asked me, ``Do you think that it is elementary that a negative multiplied by a
negative gives a positive?'' I took the bait, and said, ``Yes.'' He then told me that the great Mathematician Pierre Deligne once gave
a two--hour lecture on said question. Information Dynamics problems can feel a bit like this sometimes: beneath the seemingly--innocent,
Modal Logic lies deeeply hidden.

I will note for you that this topic is one of van Benthem's favorite things to talk about, at least from
my observations at Amsterdam. He is the creator of many of what are now known as the  `classical' problems in the field. So, I will not
try to outwit or out--write the very founder of the field. All of your reading this week will be contained in \emph{Modal Logic for
Open Minds}.

\setcounter{section}{-1}

\section{And so we beat on, boats against the current: Next week's reading}
Please read a just a bit of Chapter 12 to get an idea of the notation van Benthem uses in Chapter 15 to talk about knowledge. It is
essentially what was introduced in the `speed dating' for Epistemic logic, with the addition of an operator $C$ to indicate common
knowledge. Sections 12.1 and 12.2 should suffice to give this background. 15.1 and 15.2 should give you an idea of the kinds of problems
that can be represented and solved using Information Dynamics. I will warn that the sections following these two get a bit hairy with
the technicalities, so I leave them as optional. And having assigned you enough reading for the week, I will just ask you
to work on one problem: Please look up the Cheryl birthday problem online, and model the knowledge of the participants and the
public announcements using the methods described in Chapter 12 and Chapter 15 as a demonstration of your knowledge of the method.