\chapter[Modal logic's relationship with First-order logic]{Lost (and found) in translation: Modal logic's relationship with First-order logic}

Modal logic, as you might have suspected, is a fragment of First--order logic. The immediate observation may be that they share
all of the propositional connectives, and additonally, the accessibility relation adds a relation, requiring the power of 
First--order logic to express this. But there lies ``something deeply hidden'', as one of my mentor figures in undergrad would
have expressed it. The last week on bisimulation should have given you a feeling for this kind of relationship between
Kripke models. In the remarks following the final theorem of my notes for you, about modal equivalence and bisimulation, I
suggested that we gave something up in order to have the power of this theorem. What was this exactly? We gained much,
but we also lost some formulae along the way as we developed modal logic. Which formulae are we able to keep? They are
precisely the ones that are invariant under bisimulation. In these notes, we will explain what this means.

\setcounter{section}{-1}

\section{Standard translation of modal formulae}
We must first discuss what it means to translate a modal formula into a first--order one. This means that we are completely
eliminating $\logicbox$ and $\logicdiamond$ from our formula, and turning propostional letters $P$, $Q$ and the like into
single--place predicates that we will name things like $Px$ and $Qx$. There is a procedure for producing this kind of first--order
formula known as ``standard translation''. I will define it inductively, as I did with the definitions of the connectives and
modal operators a few weeks ago.

\begin{enumerate}
    \item Translate each propositional letter into its own single--place predicate. As mentioned above, this looks like
    turning things like $P$ and $Q$ into $Px$ and $Qx$.
    \item Translate $\perp$ into $x \neq x$.
    \item Keep in mind that statements of the form $\logicnot \phi$ can be turned into $\phi \rightarrow \perp$, and translate
    $\logicnot \phi$ into the negation of the standard translation of $\phi$.
    \item Translate $\phi \vee \psi$ into the disjunction of the standard translations of the disjuncts $\phi$ and $\psi$
    \item Translate $\logicbox \phi$ into $\forall y (Rxy \rightarrow \Phi)$, where $\Phi$ is the standard translation of $\phi$.
    \item Translate $\logicdiamond \phi$ into $\exists y (Rxy \wedge \Phi)$ where $\Phi$ is the standard translation of $\phi$. 
\end{enumerate}

After perfoming this procedure, what has happened is you have turned a modal formula into a first--order one. What did this get us?
It shows us which first--order formulae are secretly modal formulae. But one cannot always go backwards, turning all
first--order formulae into modal ones. Which are the ones that work? We will now introduce the concept of \emph{invariance
under bisimulation}.

\section{Invariance under bisimulation}
Invariance under bisimulation for a formula $\phi$ means that if two pointed Kripke models $(\mathfrak{M}, w)$ and $(\mathfrak{N}, v)$
are bisimilar, then $w \Vdash \phi$ if and only if $v \Vdash \phi$. It is an interesting procedure to prove that all
modal formulae are invariant under bisimulation, and I will send a presentation that contains this result if anyone is interested.
What is more interesting is the first--order formulae that are invariant under bisimulation. 

\section{The Big One: Van Benthem's Theorem}
This is widely considered (at least amongst the modal logicians I have encountered) to be the most beautiful theorem in all of
modal logic. My modal logic lecturer in Amsterdam told us that it was so ground--breaking when first proven and so influential
that no one ever attempted to prove it again after van Benthem did in his Ph.D. thesis. This was the first time such a result
had ever appeared in modal logic. (Additionally, this document contains the first recoreded reference to bisimuation between
Kripke models.) What van Benthem proved set the stage for the next half century or more for modal logicians. Here it is:

\bigskip

\noindent\fbox{
    \parbox{\textwidth}{
        \bigskip
       \textbf{Theorem (van Benthem, 1976)} A first--order formula $\phi(x)$ is invariant under bismulation if and only if it is equivalent to the standard translation
       of a modal formula.
       \bigskip

    }  
}
\bigskip

I will not be proving this for you, as its proof is beautful (or so I have been told), but requires much study of things that
are outside the scope of this project. The proof is given in Blackburn, de Rijke, and Venema's book \emph{Modal Logic}. The
arguments are clever and the method takes a while to get your head around. A note: these formulae are not necessarily the
ones that define the correspondences between accessibility relations and the axioms. This can be seen in the chart
on the Wikipedia page for the Kripke semantics, where some modal axioms have accessibility relations whose defining property
is not a first--order property. For example, the McKinsey axiom $\logicbox \logicdiamond \phi \rightarrow \logicdiamond \logicbox \phi$
produces a Kripke model whose accessibility relation's defining property is a secodn--order formula.

What we do get from van Benthem's theorem is a new way to look at the Kripke semantics. We see that first--order formulae that
express the relatonships between possible worlds that the Kripke semantics give us precisely the modal formualae that 
should express them.

\section{Headed into next week}
As usual, if you have gotten this far, you have nearly completed your reading for the week.
In this case, you have completed it. I have not assigned you reading from \emph{Modal Logic for Open Minds} this week
because I do not think that we have covered the topics that he finds are prerequisite to expressing the above theorem.
Translate your favorite 3 modal axioms into their standard translation. I will prepare some examples for next week.
Also, please begin to think about which of the modal logic applications you would like to explore. The choices that I am most
familiar with are the linguistic and the philosophical applications. Van Benthem covers many, many more in the book. Please
feel free to look at the table of contents in the book and explore the first paragraph of chapters that contain applications
you are interested in. I will give a ``speed--dating'' approach to the applications next week, and conclude by asking you
which of the topics you would like to explore more deeply. 
