\chapter[Our first meeting all together]{Week 0: Our first meeting all together}

\setcounter{section}{-1}
\section{Introduction of myself}
My name is Mei Rose. 
My background in Formal Logic consists mostly of undergraduate courses in Philosophical Logic, but I studied Mathematical Logic at UW and 
at the Universiteit van Amsterdam, for two and one semester respectively. I worked at a software development group in my undergraduate university 
and wrote a proof-checker in JavaScript. Modal logic has been a passion of mine since taking a course on the Philosophy of Mathematics
and seeing its history and uses since the times of Leibniz and earlier to puzzle about statements about possible worlds. My personal 
logic and mathematics heroes of history are David Hilbert and Amalie Emmy Noether. In recent history, I have looked up to Saul Aaron Kripke, 
who died in 2022 and left a wave of controversy in his wake that we will hopefully have time to discuss during this project. And alive today, 
inspiring me to be a Logician, is J. F. A. K. (Johan) van Benthem, the only person I know of who holds permanent research positions 
at three different universities at the same time.

\section{Introductions of students}
I am interested to know how your journey through mathematics and logic has been going so far. 
All I know about you is that the committee recommended you into the project, what courses you are enrolled in, 
and which courses you have taken that are relevant to the reading group. 
\begin{itemize}
    \item Please tell me about your favorite mathematical experiences. What made them special or unique?
    \item What sorts of things do you like to do outside of your coursework? Does mathematics or logic ever factor into those activities? 
    \item Who inspires you in history, or the present day?
\end{itemize}

\section{Overview and goals}
As stated in the project description, I aim for us to discuss many aspects of the so-called modal logic by the end of the semester. 
After establishing some groundwork in Kripke frames and models and a brief discussion of modal bisimulation, I plan to begin tackling 
the applications of modal logic. This is where your interests come in. I have the most experience working with modal logic 
applications in philosophy and linguistics, but there exist many more, such as information/preference theory, game theory, and more. 
Modal logic also touches more classical logic topics such as provability and model theory, and we can look into those if you are interested. 
(The ``magical'' [according to van Benthem] modal proof of Second Incompleteness is what gave this project its name.)

I would like to conclude the project with two connections between modal logic and first-order logic. 
These two connections are called van Benthem's Theorem and the Salqvist-van Benthem Algorithm. 
Each gives a slightly different viewpoint on how powerful modal logic is compared to first-order logic.

\section{Propositional and First-order logic reminders}

I will be assuming throughout this project that you are familiar with the following logical symbols from propositional logic: $\logicnot$ (logical negation,
sometimes written as $\neg$), $\wedge$ (logical conjunction), $\vee$ (logical disjunction), $\rightarrow$ (material conditional), and $\leftrightarrow$ 
(biconditional). $\phi \wedge \logicnot \psi$  $\phi \wedge \neg \psi$

$\phi \AND \psi$
