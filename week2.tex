\chapter[Modal bisimulation]{Finding common ground: Modal bisimulation an equivalence notion}

For the most part, I am going to hand this topic off to van Benthem, as he can explain it with far better examples than I can come
up with on my own. I enjoyed this section of my modal logic course in Amsterdam more than nearly anything else, if only
for the sense of humor that my lecturer brought to the topic. Looking back, I can't quite recall what was so amusing about this
topic, but it still has that sort of nostalgic charm that one feels about subjects about which one has fond memories associated.

\setcounter{section}{-1}

\section{Definition of modal bisimulation}
Modal bisimulation, in the words of van Benthem, requires ``harmony'' between two Kripke models in a way that will be clear from
the definition. My lecturer called the two conditions for the interactions of the bisimulation `back' and `forth' which I think
is appropriate. A bisimulation is a two---place relation between two Kripke models $\mathfrak{M}$ and $\mathfrak{N}$ such that
two special conditions hold.
The two conditions are the following:

\begin{enumerate}
    \item For possible worlds $w$ and $w'$, there exists a bisimulation $E$ between $\mathfrak{M}=(W, R, V)$ and $\mathfrak{N}=(W', R', V')$ if when $Eww'$,
    then $w$ and $w'$ satisfy the same propositional letters.
    \item If $Eww'$ and $Rwv$, then there is some $v' \in \mathfrak{N}$ such that $Evv'$ and $R'w'v'$. The converse of this must hold 
    as well: if $Eww'$ and $R'w'v'$, then there exists sme $v' \in \mathfrak{M}$ so that $Evv'$ and $Rwv$. 
\end{enumerate}

The first of these condition is called ``Harmony'' by van Benthem, and the second is the ``Back---and--Forth' condition. The 
first part, potentially confusingly, is the `Forth' condition, while the converse of it is the `Back' condition. The diagram 
the textbook gives makes these names a bit more clear, I think. 

\setcounter{subsection}{-1}

\subsection{Notation for bisimulations}
The bisimuation itself, the relation between sets of possible worlds in $\mathfrak{M}$ and $\mathfrak{N}$, is usually given a 
capital letter name, like $E$. It can be written as either a prefix (as I have done to keep consistent with the way that I
have been writing relations in these notes) or as an infix (as van Benthem does in our book). Essentially, it is relation
$E \subseteq W \times W'$. In the case that we are saying that two Kripke models are bisimilar (the adjective form of the word
\emph{bisimulation}), then we write $\mathfrak{M} \leftrightarroweq \mathfrak{N}$

\section{The power of bisimulation}
Bisimulation is powerful because it gives us a notion of equivalence of Kripke models. We can define another notion of 
