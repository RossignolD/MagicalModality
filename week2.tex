\chapter[Two equivalence notions]{Finding common ground: Modal bisimulation and modal equivalence}

For the most part, I am going to hand this topic off to van Benthem, as he can explain it with far better examples than I can come
up with on my own. I enjoyed this section of my modal logic course in Amsterdam more than nearly anything else, if only
for the sense of humor that my lecturer brought to the topic. Looking back, I can't quite recall what was so amusing about this
topic, but it still has that sort of nostalgic charm that one feels about subjects about which one has fond memories associated.

\setcounter{section}{-1}

\section{Definition of modal bisimulation}
Modal bisimulation, in the words of van Benthem, requires ``harmony'' between two Kripke models in a way that will be clear from
the definition. My lecturer called the two conditions for the interactions of the bisimulation `back' and `forth' which I think
is appropriate. A bisimulation is a two---place relation between two Kripke models $\mathfrak{M}$ and $\mathfrak{N}$ such that
two special conditions hold.
The two conditions are the following:

\begin{enumerate}
    \item For possible worlds $w$ and $w'$, there exists a bisimulation $E$ between $\mathfrak{M}=(W, R, V)$ and $\mathfrak{N}=(W', R', V')$ if when $Eww'$,
    then $w$ and $w'$ satisfy the same propositional letters.
    \item If $Eww'$ and $Rwv$, then there is some $v' \in \mathfrak{N}$ such that $Evv'$ and $R'w'v'$. The converse of this must hold 
    as well: if $Eww'$ and $R'w'v'$, then there exists sme $v' \in \mathfrak{M}$ so that $Evv'$ and $Rwv$. 
\end{enumerate}

The first of these condition is called ``Harmony'' by van Benthem, and the second is the ``Back---and--Forth' condition. The 
first part, potentially confusingly, is the `Forth' condition, while the converse of it is the `Back' condition. The diagram 
the textbook gives makes these names a bit more clear, I think. 

\setcounter{subsection}{-1}

\subsection{Notation for bisimulations}
The bisimulation itself, the relation between sets of possible worlds in $\mathfrak{M}$ and $\mathfrak{N}$, is usually given a 
capital letter name, like $E$. It can be written as either a prefix (as I have done to keep consistent with the way that I
have been writing relations in these notes) or as an infix (as van Benthem does in our book). Essentially, it is relation
$E \subseteq W \times W'$. In the case that we are saying that two Kripke models are bisimilar (the adjective form of the word
\emph{bisimulation}), then we write $\mathfrak{M} \leftrightarroweq \mathfrak{N}$

\section{Another notion of equivalence}
Bisimulation is powerful because it gives us a notion of equivalence of Kripke models. We can define another notion of equivalence
between Kripke models, which is called \emph{modal equivalence}. The big idea behind this is the idea of $\tau$--theories.
A $\tau$--theory of a possible world $w$ in a Kripke model $\mathfrak{M}$ the set of all $\tau$--formulae that are true in $w$.
In our case, $\tau$ is the set of symbols in the modal logic we are working in, which is the system set out by van Benthem
on the first page of Chapter 2. I use the general symbol $\tau$ here to represent the fact that construction works for any
set of modal symbols, including those that fall outside of the $\logicbox$ and $\logicdiamond$.

Two worlds $w$ and $w'$ in two Kripke models $\mathfrak{M}$ and $\mathfrak{N}$ are modally equivalent if they have the same 
$\tau$--theories. There also an idea that two Kripke models can be modally equivalent, and this in the case that the $\tau$---theories
of the \emph{models} are identical. This requires the set of formulae satisfied in all states of $\mathfrak{M}$ to be
the same as the set of all formulae satisfied in $\mathfrak{N}$. When there is a modal equivalence between two worlds,
we write that $w \leftrightsquigarrow w'$. Analogously, we write a modal equivalence between Kripke models as 
$\mathfrak{M} \leftrightsquigarrow  \mathfrak{N}$.

\section{Analogies between analogies}
We now come to the lead---up to one of the big theorems of the project. That theorem, known as van Benthem's theorem, will wait
for a few more weeks. But I will give you a taste of it with the following lead--up theorem:

\bigskip

Let $\mathfrak{M}=(W, R, V)$ and $\mathfrak{N} = (W', R', V')$ be Kripke models under a set of modalities $\tau$. Then for every $w
\in W$ and every $w' \in W'$, if $w \leftrightarroweq w'$, then $w \leftrightsquigarrow w'$. The proof is by symbolic
induction on modal formulae, and relies \emph{precisely} on each part of the definition of bisimulation.

\bigskip

This is powerful indeed! It gives us a characterisation of modal formulae, and also hints to us that bisimulation may
be giving us more than we expected. What did we sell our souls for in exchange for this epic result? That question will be
answered by the Sahlqvist--van Benthem Algorithm and van Benthem's theorem, both of which will reveal noth the expressivity and 
limitations of modal logic in the context of first--order logic.

\section{For the week ahead}
If you've made it this far in the notes, you have a basic understanding of what it means for possible worlds and Kripke
models to be bisimilar and for them to be modally equivalent. I would like you to read van Benthem's section on bisimulation,
which are contained within section 3.2 of Chapter 3. I think that with this knowledge, you should feel comfortable with
Exercises 1(a) and 1(b) that come at the end of the chapter. Feel free to create your own Kripke models and find (or fail to find)
bisimulations between them in order to get a good grasp of the concept. For fun, you may attempt to prove the above theorem
if you feel comfortable with symbolic induction.